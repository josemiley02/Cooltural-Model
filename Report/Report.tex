\documentclass{article}
\usepackage[utf8]{inputenc}
\usepackage[spanish]{babel}
\usepackage{graphicx}
\usepackage{hyperref}
\usepackage{amsmath}

\title{Análisis Científico-Técnico de un Sistema de Generación Automatizada de Guiones Culturales}
\author{Jose Miguel Leyva De la Cruz C312}
\date{}

\begin{document}

\maketitle

\begin{abstract}
Este informe evalúa un sistema computacional diseñado para generar guiones culturales optimizados, basado en métricas cuantificables como popularidad de canciones, complejidad de bailes y compatibilidad artística. Se analiza su arquitectura, modelado matemático, y contribuciones potenciales a la intersección entre ciencia de datos y arte.
\end{abstract}

\section{Introducción}
El proyecto implementa un sistema de recomendación para espectáculos culturales (teatro, música, danza) mediante técnicas de \textbf{optimización multiobjetivo} y \textbf{algoritmos evolutivos}. Su objetivo es automatizar la creación de guiones, tradicionalmente subjetiva, usando datos estructurados.

\section{Modelado del Sistema}

\subsection{Arquitectura General}
El sistema se compone de:
\begin{itemize}
    \item \textbf{Entidades}: Clases como \texttt{theatrical\_performance}, \texttt{singing\_performance}, y \texttt{dancing\_performance} que representan números artísticos.
    \item \textbf{Métricas}: Popularidad de artistas, compatibilidad género-espectáculo (matrices en \texttt{compatibilty.py}), y ratings de canciones (vía API de YouTube en \texttt{web\_rating.py}).
    \item \textbf{Algoritmo de Optimización}: Un modelo evolutivo (\texttt{model.py}) que selecciona la mejor combinación de números bajo restricciones de tiempo y diversidad.
\end{itemize}

\subsection{Modelado Matemático}
Cada número artístico calcula un \textit{rating} mediante:
\[
\text{Rating} = \sum_{i} w_i \cdot f_i(\text{métrica}_i)
\]
donde \( f_i \) son funciones de normalización (ej: promedio de popularidad de artistas) y \( w_i \) son pesos implícitos.

El algoritmo evolutivo:
\begin{enumerate}
    \item Genera una población inicial de soluciones (secuencias de números).
    \item Evalúa soluciones con una función de fitness que considera:
    \[
    \text{Fitness} = \sum \text{Ratings} - \text{Penalizaciones por repetición de tipo/artista}
    \]
    \item Aplica selección por torneo, cruce (\texttt{crossover}) y mutación para mejorar soluciones.
\end{enumerate}

\section{Aportes Científicos}

\subsection{Innovaciones}
\begin{itemize}
    \item \textbf{Matrices de Compatibilidad}: Cuantifican relaciones entre géneros artísticos y contextos (ej: "Salsa" tiene alta compatibilidad con "Festival").
    \item \textbf{Integración de Datos Externos}: Uso de APIs (YouTube) para obtener popularidad en tiempo real.
    \item \textbf{Penalizaciones Dinámicas}: Evita monotonía penalizando repeticiones de artistas o estilos en el guion.
\end{itemize}

\subsection{Relevancia}
\begin{itemize}
    \item \textbf{Industria Cultural}: Reduce la dependencia de expertos humanos en planificación de eventos.
    \item \textbf{Computational Creativity}: Demuestra que la creatividad puede modelarse mediante reglas cuantificables.
    \item \textbf{Personalización}: El sistema se adapta a distintos tipos de espectáculos (ej: "Peña" vs. "Teatro").
\end{itemize}

\section{Limitaciones y Mejoras}
\subsection{Desafíos}
\begin{itemize}
    \item \textbf{Sesgos en Datos}: Las métricas de popularidad pueden no reflejar preferencias locales.
    \item \textbf{Subjetividad Artística}: Ciertas cualidades (ej: "emocionalidad") son difíciles de cuantificar.
\end{itemize}

\subsection{Mejoras Propuestas}
\begin{itemize}
    \item Incorporar \textbf{aprendizaje automático} para ajustar pesos de métricas basado en feedback humano.
    \item Usar \textbf{NLP} para analizar letras de canciones y asignar emociones.
    \item Ampliar matrices de compatibilidad con datos antropológicos.
\end{itemize}

\section{Conclusión}
El proyecto representa un avance significativo en la automatización de procesos creativos, combinando técnicas de ciencia de datos con dominio artístico. Su enfoque híbrido (reglas explícitas + optimización) lo hace escalable y adaptable, abriendo nuevas líneas de investigación en \textbf{gestión cultural basada en datos}.

\end{document}