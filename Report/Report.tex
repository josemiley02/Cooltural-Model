\documentclass{article}
\usepackage[utf8]{inputenc}
\usepackage{graphicx}
\usepackage{amsmath}
\usepackage{hyperref}
\usepackage{indentfirst}

\title{Cooltural Model}
\author{Jos\'e Miguel Leyva De la Cruz}
\date{}

\begin{document}

\maketitle
\setlength{\parindent}{1.5em}

\section{Problema}

La primera parte de cualquier intento de soluci\'on de una problem\'atica consiste en realizar
una definici\'on formal de la misma. Por consiguiente a continuaci\'on se muestra una descripci\'on
de la misma, la cual puede ser lo m\'as cercano a una definici\'on.

\begin{center}
    \textbf{Gui\'on De Espect\'aculo}
\end{center}

Una actividad cultural, del estilo que sea, siempre va acompa\~nada de una secuencia de n\'umeros 
o actuaciones, las cuales son listadas en el famoso \textit{Gui\'on}. Siempre es un problema encontrar
la mejor secuencia que se ajusta a un Show en espec\'ifico, se deben tener en cuenta numerosos detalles,
tiempo que se espera que dure, cantidad de n\'umeros que se desean presentar y que exista una relaci\'on
entre estos, as\'i como una armon\'ia entre los mismos. 

El problema consiste en que se tienen una serie de n\'umeros culturales y se quiere confeccionar el
mejor gui\'on posible para un tipo de espect\'aculo determinado, teniendo en cuenta el tiempo de duraci\'on
de cada n\'umero y la cantidad deseada de los mismos.


\subsection*{Definici\'on Formal:}
Sea un conjunto de  $N$ n\'umeros culturales. Se desea crear una secuencia ordenada $S$ 
de dichos n\'umeros tal que se maximice el nivel de aceptaci\'on del p\'ublico.

\begin{equation*}
    max(P) = \sum_{i = 1}^{|S|} A_i
\end{equation*}

\begin{center}
    $s.a:$
\end{center}

\begin{equation*}
    \sum_{i = 1}^{|S|} T_i \leq T
\end{equation*}

\begin{equation*}
    m \leq |S| \leq M
\end{equation*}

\newpage

Donde: \begin{itemize}
    \item $A_i$ Indice de aceptaci\'on del i-\'esimo n\'umero.
    \item $T_i$ Tiempo de duraci\'on del i-\'esime n\'umero.
    \item $T$ Tiempo m\'aximo del Show
    \item $m$, $M$ Cantidad m\'inima y m\'axima de n\'umeros deseados respectivamente.
\end{itemize}

En la presente se propone un modelo matem\'atico el cual permite la resoluci\'on de 
la problem\'atica planteada.  

\section*{Importancia de Este Problema:}

El arte ha acompa\~nado a la humanidad en el camino de su evoluci\'on y desarrollo. 
En los \'ultimos tiempos se ha tratado de combinar el mundo art\'istico con el tecnol\'ogico 
y crear nuevas facilidades y habilidades para este mundillo. Los organizadores de eventos 
siempre deben tener en cuenta el desarrollo del gui\'on para poder crear un conjunto armonioso
y de gran agrado para el p\'ublico.

Un modelo que permita la creaci\'on de un gui\'on y que este tenga en cuenta la aceptaci\'on 
y satisfacci\'on del p\'ublico evitar\'ia el tedioso de organizar los n\'umeros culturales
y velar porque estos cumplan un orden l\'ogico. 

Es una oportunidad \'unica para vincular el arte con la tecnolog\'ia sin que una esfera opaque a la otra.
Adem\'as puede ser la puerta a un sin n\'umero de oportunidades y de conexiones, incluso puede aportar 
guiones m\'as creativos o dar mejores ideas que las pensadas por las personas.

\section*{Soluci\'on al Problema:}

La soluci\'on de la problem\'atica planteada se dividi\'o en dos momentos importantes, una realizaci\'on 
de la modelaci\'on y una implementaci\'on computacional para darle soluci\'on al problema. En t\'erminos
computacionales este problema puede ser descrito como un \textbf{NP-Hard}, ya que en las soluciones se 
debe tener en cuenta el orden de los conjuntos dados como respuesta. B\'asicamente dado un subconjunto
de n\'umeros culturales, cualquier permutaci\'on del mismo puede ser una coluci\'on al problema. Cabe 
destacar adem\'as que se busca maximizar, por lo que en teor\'ia todas las posibles soluciones deben
ser analizadas.

La explicaci\'on de la modelac\'on ser\'a el primer punto de esta secci\'on, y luego se explicar\'a 
brevemente el c\'odigo y los algoritmos utilizados.

\subsubsection*{Modelaci\'on:}
Se desea buscar una manera de cuantificar el nivel de aceptaci\'on del p\'ublico respecto a un n\'umero 
cultural dado, para ello lo primero es la definici\'on de la entidad \textit{N\'umero Cultural (Cultural_Number)}. Un n\'umero cultural,
de manera general cuenta con un tiempo de duraci\'on, un nombre, artistas involucrados y un tipo de manifestaci\'on. Se conoce
que las manisfestaciones art\'isticas son dis\'imiles y cada d\'ia pueden aparecer nuevas, sin embargo esta 
primera versi\'on contar\'a solamente con Canci\'on, Baile y Teatro. 
\begin{itemize}
    \item \textbf{Canci\'on (Song): } La entidad tendr\'a las mismas propiedades de N\'umero Cultural, agregando a estas
    el g\'enero de la canci\'on, si es compuesta por el artista o no, si lleva el uso de alg\'un instrumento.
    \item \textbf{Baile (Dance): } Al igual que \textbf{Song} esta entidad contar\'a con las mismas propiedades que \textbf{Cultural_Number},
    extiendose con: cantidad de bailarines, estilos de bailes implicados, canciones que se bailaran
    \item \textbf{Teatro (Theatric): } Cuenta con lo mismo que Cultural_Number
\end{itemize}



\end{document}